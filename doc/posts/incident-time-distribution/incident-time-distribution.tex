\documentclass[12pt]{article}

\usepackage[margin=1in]{geometry}
\usepackage{svg}
\usepackage{url}

\setsvg{svgpath = ../../figures/incident-time-distribution/}

\begin{document}
  \title{Fitting Incident Time to a Distribution}
  \author{Jonathan Lee}
  \date{January 27, 2016}
  \maketitle

  \section{Introduction}
    A goal of the SARBayes project is to forecast the probability of survival
    for lost persons. Researchers can use these predictions when generating
    probability maps of the lost person's location. To accomplish this, we are
    analyzing data from the International Search \& Rescue Incident Database
    (ISRID) \cite{isrid} to describe the probability of survival as a function
    of various features, such as age or temperature.

    Intuitively, a lost person should stand a better chance of survival when
    his or her absence is noticed and reported early. Therefore, one feature we
    have chosen to examine is the time at which an incident is reported—the
    incident time. Here, we present how we modeled the frequency of incidents
    as a function of the time of day.

  \section{Methods}
    \subsection{The Data}
      In ISRID, the date and time of the incident are recorded together in the
      incident date field. When the time is missing, the spreadsheet library
      reading the field provides a default time of midnight, 12:00 AM.
      Therefore, we confined our analysis to the \(N = 6356\) instances of
      incident time that were not at midnight. When these times are organized
      into half-hour bins, we can calculate the frequency of each bin and
      display the data in a histogram.

      \includesvg[width=\textwidth]{time-distribution-actual}

      The incident frequency, as shown above, roughly follows a bell-shaped
      distribution. As expected, fewer incidents are reported in the early
      hours of the morning, and a surge of incidents occurs late in the
      afternoon. The surge appears to be centered around hour 17 or
      18---roughly 5:30 PM.

    \subsection{Parameter Estimation}
      Because incident time wraps around (e.g.\ hour 0 is the same as hour 24),
      we chose to fit incident time to the circular von Mises distribution,
      which is on \([-\pi, \pi]\). This section summarizes how to fit the
      parameters of our von Mises distribution. Readers willing to take that
      step on faith may safely skip to the Results section.

      The two parameters of a von Mises distribution are \(\mu\), a real number
      describing the mean, and \(\kappa\), a positive real number describing
      the distribution's concentration. As \(\kappa\) increases, the closer the
      values cluster around \(\mu\). We converted each time on the \((0, 24)\)
      hour range to radians on \((-\pi, \pi)\), and stored the results in a
      list \(\theta\). If we think of each time as a complex number, we can
      compute an average, \(\bar{z}\), of the times.

      \[\bar{z} = \frac{1}{N} \sum_{k=1}^N (\cos \theta_k + i \sin \theta_k)\]

      \pagebreak

      According to a derivation given by the von Mises Wikipedia article
      \cite{von_mises}, the argument of \(\bar{z}\) is a biased estimator of
      \(\mu\).

      \[\mu = \arg (\bar{z})\]

      Using another equation from the article, the expectation value of the
      square of the modulus of \(\bar{z}\) is

      \[\langle |\bar{z}|^2 \rangle = \frac{1}{N} + \left(\frac{N - 1}{N}
      \right) \left(\frac{I_1(\kappa)}{I_0(\kappa)} \right)^2\]

      where \(I_\alpha\) is the modified Bessel function of order \(\alpha\).
      To estimate \(\kappa\), we iterated in increments of \(10^{-4}\) to find
      a value that best satisfied

      \[\sqrt{\left(\frac{N}{N - 1} \right) \left(|\bar{z}|^2 - \frac{1}{N}
      \right)} = \frac{I_1(\kappa)}{I_0(\kappa)}\]

  \section{Results}
    We found \(|\bar{z}|^2 = 0.0351\), \(\kappa = 0.3805\), and \(\mu =
    1.3945\). When converted back to the time of day, \(\mu\) is equivalent to
    5:20 PM. Using these parameters, we also sampled \(10^6\) values from the
    distribution. The frequency of the these values are shown below in another
    histogram. Although the curve is smoother, this fitted distribution
    approximates the actual data fairly well.

    \includesvg[width=\textwidth]{time-distribution-fitted}

  \section{Conclusion}
    The vast majority of the 80000-some cases in our copy of ISRID do not have
    an incident time. Sampling a distribution with the parameters found here
    can provide fill-in values when using incident time as a feature in the
    survival model. Probability maps can also take incident time into account,
    as lost person behavior may depend on the time of day and how much sunlight
    is available.

  \bibliography{incident-time-distribution}
  \bibliographystyle{unsrt}
\end{document}
