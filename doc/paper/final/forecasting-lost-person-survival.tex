% forecasting-lost-person-survival.tex

\documentclass[12pt,titlepage]{article}
\usepackage[margin=1in]{geometry}
\usepackage{svg}
\usepackage{url}

\def\UrlBreaks{\do\/\do-\do_}

\begin{document}
  \title{Forecasting Lost Person Survival}
  \author{Jonathan Lee}
  \date{\today}
  \maketitle

  \abstract{
    Wilderness search-and-rescue operations are often complex, expensive,
    lengthy, and not always successful. A probabilistic model of lost person
    survival over time would aid search planners when deciding how long a
    search should last. Using empirical data from the International Search \&
    Rescue Incident Database, we constructed a model using the Kaplan-Meier
    estimator to predict a given subject category's survival over time. On
    average, the absolute error of a forecast was $0.14$ from the actual
    outcome. This model demonstrated both good accuracy and stability during
    testing, and may serve as a decision-making tool for search-and-rescue
    personnel in the field.
  }

  \section{Introduction}
    \subsection{Rationale}
      When conducting search-and-rescue operations, speed is critical. Every
      passing hour allows lost persons to move ever further away from their
      last known locations along roads, trails, and waterways. As the
      potentially searchable area expands, the likelihood of a successful find
      drops, and extreme temperatures, hunger, exhaustion, and accidents result
      in fatalities. Therefore, search planners need reliable models of lost
      person motion to decide where to send their teams.

      Prolonged searches also put search teams into harm's way and strain their
      resources. In 2014, the high-profile disappearance of Malaysian Airlines
      Flight 370 and all 239 people aboard prompted an international search on
      track then to become the most expensive of its kind in aviation history
      \cite{semple}. Similarly, the costs of deploying personnel, dogs, and
      aircraft for hundreds of combined hours in a wilderness search-and-rescue
      (WiSAR) operation can quickly accumulate. Search planners must consider
      when a search no longer becomes cost-effective, and when to terminate
      that search.

      A model for describing the probability of survival over time for a given
      subject lost in the wilderness would address both of these issues.
      Researchers developing motion models may consider survival forecasts when
      predicting how lost persons behave in particular conditions, and search
      planners may examine the probability of survival to determine at what
      point the odds of a successful find are too low to continue sinking
      man-hours into a case.

    \subsection{Objective}
      The objective of this project was to describe the probability of survival
      of a lost person as a function of time, denoted here as $S(t)$, given
      information about the incident, such as subject age, category, and
      temperature. Describing survival as a probability rather than as a
      discrete status allows search planners to see a model's degree of
      confidence in the occurrence of each outcome and enables them to set
      their own criteria for discontinuing a search.

  \section{Models}
    % \ldots

    \subsection{Na{\"i}ve Survival Rate}
      This simple model predicts a subject's probability of survival is the
      same as the overall survival rate of a subset of lost persons the subject
      belongs to, regardless of how much time has elapsed. This is given by

      $$S(t) = \frac{N - d}{N} = 1 - \frac{d}{N}$$

      where $d$ is the number of deaths in the subset, and $N$ is the total
      number of subjects in the subset. Here, we separated subjects by
      category (such as hiker, hunter, or despondent), but one could also just
      as well split on age,\footnote{One could split subjects on a continuous
      variable by binning.} sex, or group size.

      When the outcomes of a subset are homogenous (that is, the majority of
      subjects either survived or died), the survival rate approaches the
      more frequent of the two outcomes. Because many categories have
      survival rates exceeding 90\%, using the survival rate predicts
      survivals well, and serves as a good baseline to compare other models to.
      In subsets more evenly split between the two outcomes, the model
      compromises by forecasting probabilities midway between dead-on-arrival
      and survival.

    \subsection{Kaplan-Meier Estimator}
      Asking for the probability of survival at a given point in time is the
      same as asking for the probability the subject will have survived at
      least until that time. The latter expression can be written as

      $$S(t) = P(T > t) = 1 - F_T(t) = 1 - \int_0^t f_T(t) dt =
        \int_t^\infty f_T(t) dt$$

      where $T$ is the random variable representing the time of death, $F_T(t)$
      is the cumulative distribution function of $T$, and $f_T(t)$ is the
      probability density function of $T$ \cite{rochford}. Given a subset of
      cases, the Kaplan-Meier estimator\footnote{The Lifelines survival
      analysis package \cite{lifelines} provided the Kaplan-Meier
      implementation we used.} approximates this true survival function
      nonparametrically as

      $$S(t) = \prod_{t_i < t} \frac{N_i - d_i}{N_i}$$

      where $t_i$ is the duration until an event---the death of a subject,
      $N_i$ is the number of subjects just before $t_i$, and $1 \leq d_i \leq
      N_i$ is the number of deaths at $t_i$. When plotted, the survival curve
      starts at $S(0) = 1$ and decreases in steps because the model uses the
      new survival rate at each death to lower the ``ceiling" on the
      probability of survival. Therefore, in theory,

      $$\lim_{t \to \infty} S(t) = 0$$

      In other words, the model assumes all subjects begin alive, but
      eventually expire.

      The Kaplan-Meier estimator also has the property of handling
      right-censoring, which occurs when we cannot determine how long survivors
      would have lived had searchers not rescued them \cite{rich}. However, in
      survival cases, the incident duration is the lowerbound of the time until
      the subject's death. The model reflects this by decreasing the $N_i$ term
      for all future events and treating survivals as subjects exiting the
      ``study."

      Like with the survival rate model, we divided subjects by category before
      fitting the curves.

  \section{Methods}
    \subsection{Data Preprocessing}
      The data used originate from the International Search \& Rescue Incident
      Database (ISRID) \cite{isrid}, which collects information about real-life
      WiSAR cases from search organizations around the world.\footnote{dbS
      Productions provided these proprietary data for free for analysis
      purposes.} Readers unconcerned with the details of our data preparation
      process may skip ahead to the evaluation procedure.

      % Refer readers to ISRID data standards?

      \subsubsection{Cleaning}
        First, to reduce the time needed to read and query the data and to
        enforce strong typing, we merged several spreadsheets containing the
        raw data together into a database backend using SQLAlchemy, a
        structured query language (SQL) abstraction toolkit \cite{sqlalchemy}.
        Rather than storing information from all incidents in one flat table,
        we defined several modular but linked tables. For instance, every
        instance of the group table pointed to one or more instances of the
        subject table. For every non-empty row, the backend attempted to
        convert each value to its column's data type. In the case of subject
        age, for example, the backend expected every value to be a positive
        real number, and attempted to extract such a number when passed a
        string.

      \subsubsection{Validation}
        To ensure the data used were of high quality, the backend checked for
        data correctness when assigning a value to a table row. For instance,
        the backend required age, weight, and incident duration to be
        nonnegative, sex to conform to one of four codes, and latitude to be
        between $-90^\circ$ and $90^\circ$. Using the key, incident number, and
        mission number attributes, we also checked for duplicate cases in the
        database, and found that all incidents were unique.

      \subsubsection{Augmentation}
        Given the time and location of an incident, we used Weather Services
        International's (WSI) online historic weather database \cite{wsi} to
        populate missing values\footnote{The backend updated a total of 2821
        instances from the weather table.} in ISRID for temperature, wind
        speed, precipitation, and solar radiation . We preferred WSI's database
        over a similar one maintained by the National Oceanic and Atmospheric
        Administration (NOAA) because measurements from the former have tended
        to align more closely with what search-and-rescue personnel on the
        ground measure \cite{lee}.

    \subsection{Evaluation}
      To measure how well each model performed on incidents it had not seen
      before, we conducted $10$-fold cross-validation with $10$ repetitions for
      each category. That is, given a subset of subjects with a common
      category of size $N$, we divided the cases into 10 contiguous segments,
      each of size $N/10$. For every segment, we trained a model on the other
      nine segments, and generated forecasts for the one segment. We repeated
      this procedure for a total of ten times and randomly shuffled the order
      of the cases between each iteration to minimize the effect of a
      fortuitous arrangement on forecast accuracy. The output of this process
      is a sequence of $10N$ forecast-observation pairs, with each observation
      represented $10$ times.

      All incidents used needed an incident duration, subject status, category,
      and group size. Then, we dropped cases where the category contained fewer
      than $20$ cases or fewer than two deaths. Without enough noncensored
      observations, the models can overfit the data by virtue of a $100\%$
      survival rate, which many of the smaller categories had. We also
      excluded cases where the subject belonged to a group because the
      curve-fitting routines can be highly sensitive to group size. The models
      are particularly prone to assigning too much weight to large blocks of
      subjects all exiting at once.\footnote{In some cases, the group received
      a single status rather than one for each individual.}

      % Also, do groups have a higher survival rate than singles?

      \subsubsection{Absolute Error}
        After generating a forecast for each observation, as described
        previously, a simple way to quantify the model's performance on that
        case is to calculate the absolute value of the difference between the
        forecasted probability and the actual outcome. Denoted here as $0 \leq
        E \leq 1$, this error is given by

        $$E = |f - o|$$

        where $f$ is the forecasted probability and $o$ is the observed outcome
        (that is, zero for dead-on-arrival\footnote{Nearly all suspended cases
        result in deaths, and so we treated them as such.} and one for
        survival). As the forecast becomes more accurate, the absolute error
        approaches zero.

        $$\lim_{f \to o} E = 0$$

      \vfill

      \subsubsection{Brier Score}
        A Brier score\footnote{This scoring rule is essentially the mean of
        the square of every error term.} is another scoring measure for
        assessing probabilistic forecasts, given by

        $$BS = \frac{1}{N} \sum_{i=1}^N {E_i}^2 = \frac{1}{N}
          \sum_{i=1}^N (f_i - o_i)^2$$

        where $f_i$ is the forecasted probability of the $i$th case, $o_i$ is
        the observed outcome, and $N$ is the number of cases evaluated
        \cite{brier+score}. As forecasts become more accurate, the Brier score
        approaches zero.

  \section{Results}
    % The title of each plot already denotes a new section.
    % Subsections would be redundant.

    \setsvg{svgpath=../../figures/kaplan-meier/}

    \includesvg[width=\textwidth]{grid}

    Kaplan-Meier curves are shown above for the nine most common categories.
    The $95\%$ credence intervals are given by the shaded regions. Survivals,
    which are instances of right-censoring, are indicated by the tick marks on
    the curves. The number of observations is denoted in the title of each plot
    as $N$.

    The curve for despondents drops off especially sharply. After just over
    a day has passed, the odds of survival are worse than even.

    \setsvg{svgpath=../../figures/evaluation/}

    \includesvg[width=\textwidth]{abs-error-diff-dist}

    Using the cross-validation procedure described previously, for every
    incident, we took the difference between the absolute error of the survival
    rate model's forecast and the absolute error of the Kaplan-Meier model's
    forecast.

    $$\Delta E = E_{rate} - E_{km} = |f_{rate} - o| - |f_{km} - o|$$

    The distribution of the error differences are charted in the $200$-bin
    histogram\footnote{The frequency scale has been adjusted downwards by a
    factor of $10$ to account for the cross-validation repetitions so that the
    sum of the bar heights equals $N$, the total number of observations in all
    included categories.} shown above. The blue dashed line at zero is where
    the models perform equally well, and the area between the red dashed lines
    at $-0.05$ and $0.05$ represents a ``null" zone where the error difference
    is negligable.

    A positive difference indicates the Kaplan-Meier model's forecast had an
    error smaller than that of the time-blind survival rate model, which serves
    as a benchmark. In about half of the cases,\footnote{The exact proportion
    may vary because of the random shuffling.} the Kaplan-Meier model
    exhibited significant improvement over its na{\"i}ve counterpart. In about
    a fifth of the cases, the Kaplan-Meier performed significantly worse. The
    remainder fall inside the ``null" zone. For a search planner, the fat right
    tail of the distribution means that using the Kaplan-Meier model over a
    general survival rate can translate into a $0.1$ gain in raw accuracy or
    greater for a sizeable number of cases.

    \includesvg[width=\textwidth]{brier-score-boxplot}

    Within each category, we took the Brier score of the forecast-observation
    pairs, and plotted a boxplot of the scores of each model. The size of each
    point represents the relative size of the category. Points representing the
    categories with the highest survival rates received the strongest shades of
    green, while the reddest points represented categories with the lowest
    survival rates. The whiskers represent the extremes of the scores, and the
    box represents the quartiles.

    For all categories examined, the Brier scores from the Kaplan-Meier
    estimator were less than or equal to those from the survival rate model,
    indicating modest improvement for some categories and none for others.
    Unsurprisingly, categories with lower survival rates and a more even split
    between the outcomes exhibited greater Brier scores in both models.
    However, such low-survival categories appeared to have benefited the most
    from the Kaplan-Meier model's forecasts, as the maximum score decreased
    more than the minimum score decreased between the two plots.

    \includesvg[width=\textwidth]{brier-score-comparison}

    On this plot, each point represents a category, and how well each model
    scores relative to the other determines the point's position. Points
    representing categories where the models perform equally well should fall
    along the dashed blue line. The point will fall left of the line if the
    Kaplan-Meier model outperforms the survival rate model, and will fall below
    the line if the Kaplan-Meier model underperforms. Like with the boxplot,
    the size and color of each point represent that category's relative size
    and survival rate, respectively.

    As previously noted, many of the low-survival categories exhibited the
    greatest gains in forecast accuracy. On an absolute scale, both models
    display relatively good performance. For comparison, a completely
    uninformed model guessing a $0.5$ probability of survival on every case
    would always attain an absolute error of $0.5$. This would result in a
    Brier score of $0.25$, which is greater than the worst score for either
    model.

  \section{Conclusion}
    \subsection{Accuracy}
      The Kaplan-Meier estimator matches the survival rate model's strong
      performance among high-survival categories while improving on the
      accuracy of predictions for low-survival categories. In terms of absolute
      error, on average, the former's forecasted probabilities miss the actual
      outcome of an incident by $0.14$, while the latter's are off by $0.19$.

    \subsection{Robustness}
      For both models, the Brier scores and mean absolute errors were nearly
      identical between cross-validation runs, indicating a high degree of
      stability. The Kaplan-Meier estimator is an especially good empirical fit
      for the data because it only uses the distributions of deaths and
      survivals through time to make its forecasts.

      However, for the Kaplan-Meier model, the number of subjects remaining in
      the population will inetivably shrink as time passes, leading to
      overfitting. For instance, for female hikers, the subject with the
      longest incident duration of $13$ days happened to be dead-on-arrival,
      causing the model to drop abruptly from a one in three chance of survival
      to zero. A search planner would do best to be skeptical of long-term
      projections where cases are sparse.

    \subsection{Future Work}
      While we sorted subjects by category, there may be merit in slicing the
      incidents in other ways. In particular, comparing subjects lost alone
      against subjects lost in groups, which were excluded here, may give
      search planners insight into how survival differs between the two
      populations. One could also use the Kaplan-Meier curves found here to
      calculate hazard ratios for different categories relative to a control
      population. This would give search planners empirical data on which
      categories have the worst survival rates, which could be useful for
      decision-making purposes.

    % \footnote{\url{https://github.com/jonathanlee1/SARBayes}}

  \bibliographystyle{unsrt}
  \bibliography{forecasting-lost-person-survival}
\end{document}
