% forecasting-lost-person-survival.tex

\documentclass[12pt,titlepage]{article}
\usepackage[margin=1in]{geometry}
\usepackage{svg}
\usepackage{url}

\begin{document}
  \title{Forecasting Lost Person Survival}
  \author{Jonathan Lee}
  \date{\today}
  \maketitle

  % \abstract{
  % }

  \section{Introduction}
    \subsection{Rationale}
      When conducting search-and-rescue operations, speed is critical. Every
      passing hour allows lost persons to move ever further away from their
      last known locations along roads, trails, and waterways. As the
      potentially searchable area expands, the likelihood of a successful find
      drops, and extreme temperatures, hunger, exhaustion, and accidents result
      in fatalities. Therefore, search planners need reliable models of lost
      person motion to determine where to send their teams.

      Prolonged searches also put search teams into harm's way and strain their
      resources. In 2014, the high-profile disappearance of Malaysian Airlines
      Flight 370 and all 239 people aboard prompted an international search on
      track to become the most expensive of its kind in aviation history
      \cite{semple}. Similarly, the costs of deploying personnel, dogs, and
      aircraft for hundreds of combined hours in a wilderness search-and-rescue
      (WiSAR) operation can quickly accumulate. Search planners must consider
      when a search no longer becomes cost-effective, and when to terminate
      that search.

      A model for describing the probability of survival over time for a given
      subject lost in the wilderness would address both of these issues.
      Researchers developing motion models may consider survival forecasts when
      predicting how lost persons behave in particular conditions, and search
      planners may examine the probability of survival to determine at what
      point the odds of a successful find are too low to continue sinking
      man-hours into a case.

    \subsection{Objective}
      The objective of this project was to describe the probability of survival
      of a lost person as a function of time, denoted here as $S(t)$, given
      information about the incident, such as subject age, category, and
      temperature. Describing survival as a probability rather than as a
      discrete status allows search commanders to see a model's degree of
      confidence in the occurrence of each outcome and to adjust their plan
      accordingly.

  \section{Methods}
    \subsection{Data}
      The data used originate from the International Search \& Rescue Incident
      Database (ISRID) \cite{isrid}, which collects information about real-life
      WiSAR cases from search organizations around the world. dbS Productions
      provided these proprietary data for free for analysis purposes.

      \subsubsection{Cleaning}

      \subsubsection{Validation}

      \subsubsection{Augmentation}

      % Because nearly all suspended cases result in dead-on-arrival,
      % Other statuses, such as injured, were treated as survivals

      %\subsubsection{Migration}
      %  First, to enable timely retrieval, querying, and strict typing, we
      %  constructed a database backend using SQLAlchemy, a structured query
      %  language (SQL) abstraction toolkit \cite{sqlalchemy}. The script
      %  merging together the spreadsheets into a single SQLite file also stored
      %  information about each incident across several independent tables,
      %  rather than in a single flat table.

      %\subsubsection{Cleaning}
      %  For every value extracted from each spreadsheet, we attempted to
      %  convert that value into the type required by the corresponding column.
      %  If the conversion failed, we stored that value in a pickled Python
      %  dictionary. Later, empty rows were removed.

      %\subsubsection{Validation}
      %  For columns that should have positive values, such as age or weight,
      %  we used SQLAlchemy's built-in validation procedures to ensure the
      %  correctness of values upon assignment. Using the key, incident number,
      %  and mission number, the unit tests for the database infrastructure also
      %  determined there were no duplicate cases.

      %\subsubsection{Augmentation}
      %  Given a time and coordinates as latitude and longitude, we used Weather
      %  Services International's (WSI) online historic weather database
      %  \cite{wsi} to populate missing values in ISRID for temperature, wind
      %  speed, and precipitation.

    \subsection{Models}
      \subsubsection{Na{\"i}ve Survival Rate}
        This simple model predicts a subject's probability of survival is the
        same as the overall survival rate of the subset of lost persons the
        subject belongs to, regardless of how much time has elapsed. This is
        given by

        $$S(t) = \frac{N - d}{N} = 1 - \frac{d}{N}$$

        where $d$ is the number of deaths in the subset, and $N$ is the total
        number of subjects in the subset.

        When the outcomes of a subset are homogenous (that is, the majority of
        subjects either survived or died), the survival rate approaches the
        more frequent of the two outcomes. Because many categories have
        survival rates exceeding 90\%, using the survival rate predicts
        survivals well, and serves as a good baseline for comparing other
        models to. Conversely, in subsets more evenly split between the two
        outcomes, this model compromises by forecasting probabilities midway
        between dead-on-arrival and survival.

      \subsubsection{Kaplan-Meier Estimator}
        Asking for the probability of survival at a given point in time is the
        same as asking for the probability the subject will have survived at
        least until that time. The latter expression can be written as

        $$S(t) = P(T > t) = 1 - F_T(t) = 1 - \int_0^t f_T(t) dt = \int_t^\infty f_T(t) dt$$

        where $T$ is the random variable representing the the time of death,
        $F_T(t)$ is the cumulative distribution function of $T$, and $f_T(t)$
        is the probability density function of $T$ \cite{rochford}. The Kaplan-
        Meier estimator approximates this true survival function
        nonparametrically by setting the survival rate at each death as the new
        ``ceiling" on the probability of survival. Given a subset of similar
        cases, the survival function is estimated as

        $$S(t) = \prod_{t_i < t} \frac{N_i - d_i}{N_i}$$

        where $t_i$ is the duration until an event---the death or withdrawal of
        a subject, $N_i$ is the number of subjects at $t_i$, and $0 \leq d_i
        \leq N_i$ is the number of deaths at $t_i$.

        %When plotted, the survival curve starts at $S(0) = 1$ and then
        %decreases in a series of steps so that

        %$$\lim_{t \to \infty} S(t) = 0$$

        %In other words, the model assumes all subjects begin alive, but
        %eventually expire.

        %Also, the Kaplan-Meier esimator handles right-censoring, when
        %subjects are rescued and we cannot see when they would have died, had
        %searchers not found them.

  \section{Results}
    \subsection{Plots}
      \setsvg{svgpath=../../figures/kaplan-meier/}
      \includesvg[width=\textwidth]{km-single-subject-grid}

      Above are Kaplan-Meier curves for the nine most common categories,
      excluding subjects traveling in groups. The $95\%$ credence intervals are
      given by the shaded regions. Survivals, which are right-censored events,
      are indicated with tick marks.

    \subsection{Evaluation}
      \setsvg{svgpath=../../figures/}
      \includesvg[width=\textwidth]{brier-score-boxplot}
      \includesvg[width=\textwidth]{brier-score-comparison}
      \includesvg[width=\textwidth]{pos-abs-error-diff-dist}

  \section{Conclusion}
    \ldots
    % Good empirical fit



    %\subsection{Future Work}
    %  Future work may focus using external geographic data to augment other
    %  fields in the database, such as elevation change or track offset.

  \bibliographystyle{unsrt}
  \bibliography{forecasting-lost-person-survival}
\end{document}
