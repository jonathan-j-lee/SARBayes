% poster.tex

\documentclass[serif,final]{beamer}
\usepackage[scale=1.6]{beamerposter}
\usepackage{multicol}
\usepackage{ragged2e}
\usepackage{svg}

\usetheme{Berlin}
\usecolortheme{dove}
\beamertemplatenavigationsymbolsempty

\setlength{\columnsep}{1in}
\setbeamersize{text margin left=1.5in, text margin right=1.5in}
\newcommand{\heading}[1]{{\large \textbf{#1}}\\}
\newcommand{\subheading}[1]{{\textbf{#1}}\\}

\title{Forecasting Lost Person Survival}
\author{Jonathan Lee}
\institute{TJHSST Computer Systems Lab}
\date{\today}

\begin{document}
  \justify

  \begin{frame}
    \titlepage

    \begin{multicols}{3}
      \heading{Introduction}

      \indent

      \subheading{Purpose}

      When conducting wilderness search-and-rescue (WiSAR) operations, speed is
      critical. Every passing hour allows lost persons to move ever further
      away from their last known locations along roads, trails, and waterways.
      As the potentially searchable area expands, the likelihood of a
      successful find drops, and extreme temperatures, hunger, exhaustion, and
      accidents result in fatalities. Therefore, search planners need reliable
      models of lost person motion to determine where to send their teams.
      Prolonged searches also put search teams into harm's way and strain their
      resources.

      \indent

      A model for describing the probability of survival over time for a given
      subject lost in the wilderness would help address both of these issues.
      Researchers developing motion models may consider survival forecasts when
      predicting how lost persons behave in particular circumstances, and
      search planners may examine the probability of survival to determine when
      the odds of a successful find are too low to continue searching.

      \indent

      \subheading{Objective}

      The objective of this project was to describe the probability of survival
      as a function of time, denoted here as $S(t)$, given a list of attributes
      about the incident, such as subject age, category, and temperature.

      \indent

      $S(t)$ should have the following properties:

      $$S(0) \approx 1, \lim_{t \to \infty} S(t) = 0$$

      In other words, we assume nearly all subjects start out alive, but
      eventually expire.

      \indent

      \vfill
      \columnbreak

      \heading{Methods}

      The data used originate from the International Search \& Rescue Incident
      Database (ISRID) \cite{isrid}, which collects information about real-life
      WiSAR cases.

      \indent

      \subheading{Model}

      Asking for the probability of survival at a given time is the same as
      asking for the probability the subject will have survived at least until
      that time. In survival analysis, the latter can be written as

      $$S(t) = P(T > t) = 1 - F_T(t) = \int_t^\infty f_T(t) dt$$

      where $T$ is the random variable representing the time of death, $F_T(t)$
      is the cumulative distribution function of $T$, and $f_T(t)$ is the
      probability density function of $T$ \cite{rochford}. To approximate the
      true survival function, we used the Kaplan-Meier estimator, given by

      $$S(t) = \prod_{t_i < t} \frac{N_i - d_i}{N_i}$$

      where $t_i$ is the time of every event before $t$, $N_i$ is the number of
      subjects just before $t_i$, and $0 \leq d_i \leq N_i$ is the number of
      deaths at $t_i$.

      \indent

      As shown, the Kaplan-Meier estimator uses the cases themselves to
      calculate an empirical probability of survival, which decreases over
      time. Such a nonparametric model can adapt to new data well. Also, the
      Kaplan-Meier esimator handles right-censoring, which occurs when subjects
      are rescued and we cannot see when they would have died, had searchers
      not found them.

      \indent

      \vfill
      \columnbreak

      \heading{Results}

      \begin{center}
        \setsvg{svgpath=../figures/kaplan-meier/}
        \includesvg[width=0.32\textwidth]{grid-large}
      \end{center}

      The Kaplan-Meier curves for the four most common categories, excluding
      subjects in groups, are shown above. The $95\%$ credence intervals are
      given by the shaded regions. Survivals are indicated by the tick marks.

      \indent

      \heading{Conclusion}
        As shown, Kaplan-Meier curves fit the data well, and are robust enough
        to handle noisy data. Unlike other models, like logistic curves, the
        Kaplan-Meier estimator can capture a range of survival processes well.

      \indent

      \heading{References}
      \bibliographystyle{unsrt}
      \bibliography{../paper/final/forecasting-lost-person-survival.tex}
    \end{multicols}
  \end{frame}
\end{document}
